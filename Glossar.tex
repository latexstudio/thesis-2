% http://mirrors.ibiblio.org/CTAN/macros/latex/contrib/glossaries/glossariesbegin.pdf
% http://mirrors.ibiblio.org/CTAN/macros/latex/contrib/glossaries/glossaries-user.pdf

\usepackage[
    acronym,
    %nomain,
    nonumberlist,
    nopostdot,
    shortcuts,
    smallcaps,
    toc,
    xindy,
]{glossaries}

\makeglossaries

% Usage: \gls{foo} -> Will automatically place long version first once, then short.
% Uses Small caps.
\setacronymstyle{long-sc-short}
\setglossarystyle{list}

\newacronym{http}{Http}{Hyper Text Transfer Protocol}
\newacronym{mpeg-1}{Mpeg-1}{Moving Picture Experts Group Phase 1}
\newacronym{nat}{Nat}{Network Address Translation}
\newacronym{www}{Www}{World Wide Web}
\newacronym{nes}{Nes}{Nintendo Entertainment System}
\newacronym{snes}{Snes}{Super Nintendo Entertainment System}
\newacronym{rom}{Rom}{Read-Only Memory}
\newacronym{gui}{Gui}{Graphical User Interface}

\newglossaryentry{accessibility}{name={Accessibility},description={TBD}}
\newglossaryentry{canvas}{name={Canvas},description={TBD}}
\newglossaryentry{controller}{name={Controller},description={Gerät zur Steuerung von Computerspielen. Bei alten Konsolen über Kabel verbunden}}
\newglossaryentry{django}{name={Django},description={Web-Framework für die Programmiersprache Python}}
\newglossaryentry{docker}{name={Docker},description={Plattform zur Erstellung und Verwaltung von Software-Containern}}
\newglossaryentry{emulator}{name={Emulator},description={Computer-Programm zur Emulation von Konsolenspielen}}
\newglossaryentry{host}{name={Host},description={Dienstanbietender Teilnehmer in einem Computer-Netwerk}}
