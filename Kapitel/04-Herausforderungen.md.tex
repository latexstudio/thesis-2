\chapter{Probleme \&
Lösungsansätze}\label{probleme-luxf6sungsansuxe4tze}

\blindtext[1]

\section{Web-Applikation in Echtzeit}\label{web-applikation-in-echtzeit}

\begin{itemize}
\tightlist
\item
  Stateless Request/Response-Modell macht aufgrund der Anforderungen
  keinen Sinn
\item
  Man könnte selektiv z.B. AJAX / Long Polling etc. machen. Macht aber
  wenig Sinn, weil man, wie man später sieht, WebRTC braucht und man
  dafür einen Signaling Channel bauen muss. Und da bietet sich
  WebSockets an.
\end{itemize}

\section{Emulation im Web}\label{emulation-im-web}

\begin{itemize}
\tightlist
\item
  Informationsaustausch zwischen Web-Browser und Emulator
\end{itemize}

\section{Transport \& Synchronisation der
Spieldaten}\label{transport-synchronisation-der-spieldaten}

\begin{itemize}
\tightlist
\item
  Server-Client-Modell, wo die Emulation auf einem Game-Server statt
  findet. Vor/Nachteile
\end{itemize}

\section{Kommunikation zwischen Browser und
Controller-App}\label{kommunikation-zwischen-browser-und-controller-app}
