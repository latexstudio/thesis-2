\chapter{Lösungsansätze}\label{luxf6sungsansuxe4tze}

\blindtext[1]

\section{Vorhandene Teillösungen}\label{vorhandene-teilluxf6sungen}

\subsection{melody-jsnes}\label{melody-jsnes}

\blindtext[1]

\subsection{webtendo}\label{webtendo}

\blindtext[1]

\subsection{retroarch-web}\label{retroarch-web}

\begin{itemize}
\tightlist
\item
  Zu kompliziert, snes9x nur ein Emulator unter vielen

  \begin{itemize}
  \tightlist
  \item
    Nicht benutzerfreundlich
  \end{itemize}
\item
  Reagiert nicht auf Touch-Events
\end{itemize}

\blindtext[1]

\section{Lösungsansatz 1:
Game-Server}\label{luxf6sungsansatz-1-game-server}

Definition list

Is something people use sometimes.

Markdown in HTML

Does \emph{not} work \textbf{very} well. Use HTML tags.

\begin{description}
    \item [Pro] Client-Seite sehr einfach und portabel (jsmpeg spielt den mpeg-ts stream [mpeg1/mp2] ab und fertig).
    \item [Pro] Super Browser-Unterstützung. Habe keinen Browser gefunden, der den Stream nicht abgespielt hat.
    \item [Contra]  Minimaler Prototyp (mit statischem Video) hat nur funktioniert, in dem Audio und Video getrennt kodiert und zum jsmpeg-Websocket-Relay geschickt wurde.
\item [Contra] Der Quellcode des Emulator(frontend)s müsste angepasst und erweitert werden:
    \item [Contra] retroarch-ffmpeg unterstützt Live-Streaming und es sind auch verschiedenste Optionen konfigurierbar. Problem: mpeg1/mp2 wird nicht unterstützt, zumindest habe ich keine funktionierende Konfiguration gefunden. Für die Unterstützung von mpeg1/mp2 müsste also der retroarch-ffmpeg-core angepasst werden (viel C, wenig Kommentare). Ist prinzipiell machbar, aber nicht mit meiner C-Expertise, und nicht innerhalb des gegebenen Zeitrahmens.
    \item [Contra] Auch schwierig: Die Inputs müssten von den Clients an retroarch geschickt werden. Hier müsste wieder das Emulator-Frontend (retroarch) angepasst werden, C/C++. Man hätte wahrscheinlich einen Websocket-Server bauen können, der dann Nachrichten mit Payload=Controller\-/Input enthält und diese Nachrichten dann auf retroarch\-/Controller\-/Eingaben mappt und an das Emulator\-/Frontend weitergibt.
    \item [Contra] ffmpeg, libav, Encodings, (spezielles) Muxing, \mbox{A/V\-/Sync}, Doku jeweils katastrophal
\end{description}

Noch zu testen: Paralleles Encoden von Audio/Video mit retroarch-ffmpeg.
Das war für jsmpeg sowieso nötig.

\subsection{Vorteile und Nachteile}\label{vorteile-und-nachteile}

\section{Lösungsansatz 2: Emulation im
Browser}\label{luxf6sungsansatz-2-emulation-im-browser}

\begin{itemize}
\tightlist
\item
  Warum sind die Smartphones mit dem ``eigenen Bildschirm'' verbunden,
  und nicht mit dem Host-PC?

  \begin{itemize}
  \tightlist
  \item
    Minimierung des lokalen Input-Lags zwischen Anzeige und Steuerung
  \end{itemize}
\end{itemize}

Tamarys WebRTC-Plugin

\subsection{Vorteile und Nachteile}\label{vorteile-und-nachteile-1}

\section{Zusammenfassung}\label{zusammenfassung}

\blindtext
