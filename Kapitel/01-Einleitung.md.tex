\setchapterpreamble[uc][.75\textwidth]{%
\dictum[Lewis Carroll, \textit{Alice in Wonderland}]{%
``Begin at the beginning,'' the King said, gravely, ``and go on till you come to an end; then stop.''}\vskip1em}

\chapter{Einleitung}\label{einleitung}

Nach der Krise der Videospiel-Industrie Mitte der achtziger Jahre hatte
Nintendo mit der Veröffentlichung des \gls{nes} eine Konsole für
Videospiele auf den Markt gebracht, die sich millionenfach verkaufte.
Dabei führte das Unternehmen weltbekannte Marken wie Mario, Donkey Kong
und Zelda ein, mit denen das Unternehmen bis zum heutigen Tag
erfolgreich ist. In den 90er Jahren wurde der Nachfolger des \gls{nes}
auf den Markt gebracht, das \gls{snes}. Beide Systeme verkauften sich
zusammen weltweit über 100 Millionen mal und sind damit die
erfolgreichsten Systeme ihrer Generation. Insgesamt wurden für beide
Systeme ca. 1495 Spiele entwickelt und lizensiert; darunter
Singleplayer-Klassiker wie Super Metroid oder \enquote{Mega Man}.
Gespielt werden konnte aber nicht nur allein: Die Konsolen wurden von
vornherein für mehrere Spieler ausgelegt und verfügen über mehrere
Controller-Anschlüsse zur Eingabe zur Spielsteuerung. Dies führte dazu,
dass viele Spiele sowohl einen Singleplayer- als auch einen
Multiplayer-Modus beinhalteten, der es ermöglichte, mit mehreren
Spielern an der Konsole zu spielen.

Mittlerweile haben sich mindestens zwei weitere erfolgreiche
Videospiel-Konsolen im Markt etabliert: Die Xbox One von Microsoft und
die PlayStation 4 von Sony. Wie auch schon das \gls{nes} unterstützen
beide Systeme mehrere Spieler über Controller, die mit der Konsole
gekoppelt sind. Die technologische Entwicklung seit den 80er und 90er
Jahren ist gut am Internet abzulesen: Üblich waren Internetanschlüsse
mit einer Bandbreite von weit unter einem Megabit.
\todo{Formulierung und Quelle}Heutzutage sind breitbandige
Internetanschlüsse großflächig verfügbar. Dieser technologische
Fortschritt ermöglicht es, größere Datenmengen übers Internet zu
übertragen -- z. B. die Daten eines Multiplayer-Spiels. Man spielt nicht
mehr exklusiv im eigenen Wohnzimmer oder auf einer LAN-Party gemeinsam
miteinander, sondern kann dies über das Internet und über Landesgrenzen
hinweg tun. Dazu haben Microsoft und Sony jeweils kostenpflichtige
Online-Plattformen geschaffen, über die Spieler miteinander über
verschiedenste Kanäle interagieren -- und, viel wichtiger -- miteinander
spielen können. Die Plattformen bieten einen Treffpunkt der
Kommunikation und Vermittlung und stellen für die Spiele-Entwickler
Dienste wie Spiel-Lobbys und Matchmaking bereit.

Eine bemerkenswerte Eigenschaft beider Plattformen ist deren einfache
Handhabung: Sie machen es den Spielern sehr einfach, gemeinsam über das
Internet miteinander zu spielen. Die Plattformnutzer müssen über keine
speziellen Kenntnisse verfügen, müssen keine zusätzliche Software
installieren oder etwa darin geübt sein, einen dedizierten Game-Server
aufsetzen und zu administrieren. Mitglieder der Plattform können sich
auf das Spielen konzentrieren. Aktuelle Spiele-Titel können ohne große
Hürden über die beschriebenen Plattformen gespielt werden. Eine
Plattform für \gls{snes}-Spiele, die einen vergleichbaren Komfort
bietet, existierte bislang nicht.

\section{Motivation}\label{motivation}

Möchte man am PC ein Konsolenspiel, z. B. Super Mario, spielen, benötigt
man neben dem eigentlichen Spiel auch ein Programm, mit dem das Spiel
ausgeführt werden kann: einen Emulator. Einen geeigneten Emulator
auszuwählen ist nicht einfach: Für das \gls{snes} allein existieren mehr
als ein Dutzend Emulatoren für verschiedene Betriebssysteme (vgl.
\cite{snes-emulator-list}), die sich in der Genauigkeit der Emulation,
der Kompatibilität und weiteren Aspekten unterschieden und von denen
partiell zusätzliche Ableger (Forks) existieren\todo{}. Ein Teil dieser
Emulatoren verfügt über einen Multiplayer-Modus. Mit dem Problem, dass
die verschiedenen Emulatoren untereinander meist inkompatibel
sind\todo{}. Auch bei komplexeren Emulationssystemen wie RetroArch, das
viele Emulatoren enthält und um eine eigens geschaffene Netplay-Funktion
erweitert, ist die Kompatibilität nicht garantiert (vgl.
\cite{retroarch-netplay-comp}). Um erfolgreich ein \gls{snes}-Spiel im
Multiplayer-Modus gespielt werden, müssen alle Spieler im Idealfall den
gleichen Emulator in der selben Version verwenden. Sofern die
verwendeten Emulatoren unteinander kompatibel sind, existieren weitere
Hürden, die für einen Anwender ohne technischen Hintergrund nicht ohne
Weiteres zu lösen sind. Ein möglicher Ablauf zum Erstellen eines Spiels
sieht beispielsweise so aus:

\begin{enumerate}
\def\labelenumi{\arabic{enumi}.}
\tightlist
\item
  Einer der Spieler wird als Spielleiter (\gls{host}) bestimmt und
  startet den Emulator im Netzwerk-Modus.
\item
  Der Host muss dabei sicherstellen, dass der spezifische Port auf dem
  der Emulator-Prozess läuft von allen Mitspielern über das Netzwerk
  erreichbar ist. Falls der Host-Spieler sich in einem Netzwerk
  befindet, dessen Routing auf \gls{nat} basiert, muss eine
  Port-Weiterleitung für den Emulator-Prozess eingerichtet werden, der
  auf dem PC des Host-Spielers läuft. Dazu muss die verwendete
  Portnummer des Emulators bekannt und eine Port-Weiterleitung im
  Administrationsbereich des Routers konfiguriert sein. Für das
  beschriebene Szenario ist die korrekte \gls{nat}-Konfiguration
  Voraussetzung für das Spielen im Multiplayer-Modus.
\item
  Wenn der ausgewählte Emulator über keine Funktion verfügt, um
  Netzwerkspiele aufzulisten, benötigen alle Mitspieler die IP-Adresse
  des Host-Spielers, um eine Verbindung herzustellen.
\item
  Der Host-Spieler muss die eigene öffentliche IP-Adresse ermitteln und
  seinen Mitspielern mitteilen.
\item
  Die Mitspieler verbinden sich mit dem Host-Spieler über die Eingabe
  der IP-Adresse des Hosts im entsprechenden Dialog der
  Benutzeroberfläche des verwendeten Emulators.
\end{enumerate}

\section{Zielsetzung}\label{zielsetzung}

Im Zuge der immer besseren Verfügbarkeit von breitbandigen
Internetanschlüssen entstehen neue Chancen für die Entwicklung neuer und
interessanter Lösungen. Mit den entstandenen Chancen ergeben sich aber
auch neue Herausforderungen, für deren Lösung neue Technologien
geschaffen werden müssen. Ein technoligischer Bereich, der sich
besonders schnell weiterentwickelt, ist das das \gls{www}, das
spätestens seit Beginn der Web 2.0-Ära einen riesen Aufschang erfährt
und sich die Computer-Nutzung immer mehr in den Internet-Browser und das
\gls{www} verlagert.

Ablesen lässt sich die Weiterentwicklung der Web-Technologien zum
Beispiel durch die große Anzahl der neuen Standards- und Protokolle. Der
Internet-Browser verfügt nahezu über den Funktionsumfang eines
Desktop-Systems und kann ein breites Spektrum von Anwendungsfällen
abdecken. Das Abspielen von Audio- und Videodaten, die direkte
Kommunikation zwischen Browsern und das Ausführen von komplexen
3D-Anwendungen ist mittlerweile kein Problem mehr.

Diese Arbeit untersucht die Frage, ob die aktuell verfügbaren
Web-Standards die Anforderungen für die Entwicklung einer
Multiplayer-Plattform für \gls{snes}-Spiele erfüllen und welche
Technologien in welcher Weise kombiniert werden müssen, um ein
funktionales Basis-System zu schaffen. Die zentralen Aspekte sind die
Auswahl, die Kombination und die Integration der verschiedenen
Technologien zu einem Gesamtsystem bestehend aus verschiedenen
Komponenten, die alle im Zuge dieser Arbeit entwickelt werden.

Der Fokus liegt dabei auf der Entwicklung einer funktionalen
Web-Plattform mit den wichtigsten Basisfunktionen, die aufgrund einer
sauberen Architektur leicht um neue Funktionen zu erweitern ist.

\section{Vorgehensweise}\label{vorgehensweise}

\begin{enumerate}
\def\labelenumi{\arabic{enumi}.}
\tightlist
\item
  \textbf{Grundlagen} Einarbeitung und Erläuterung der technischen
  Grundlagen, die für die Lösung der Zielsetzung notwendig sein können.
  Dies umfasst verschiedene Web-Technologien auf der einen Seite und
  Emulationssysteme auf der anderen Seite.
\item
  \textbf{Spezifikation} Genaue Festlegung des Funktionsumfangs des zu
  erstellenden Systems.
\item
  \textbf{Herausforderungen} Ableitung der zu lösenden Teilprobleme, die
  sich aus den definierten Zielen ergeben.
\item
  \textbf{Lösungsansätze} Untersuchen, ob und welche Teillösungen für
  die im vorherigen Abschnitt beschriebenen Probleme existieren.
  Vorstellen von Lösungsansätzen, die ggf. auf vorhandenen Teillösungen
  aufbauen oder diese erweitern. Gewählten Lösungsansatz vorstellen und
  mögliche Alternativen aufzeigen.
\item
  \textbf{Methodik} Beschreibung der geplanten Umsetzung.
\item
  \textbf{Konzeption} Entwurf der Benutzeroberfläche, Definition der
  System-Komponenten und der Software-Architektur.
\item
  \textbf{Implementierung} Technische Beschreibung der entwickelten
  Lösung mit Fokus auf ausgewählte Aspekte der Implementierung.
\item
  \textbf{Evaluation} Überprüfung und Bewertung der Lösung im Hinblick
  auf das Erreichen der definierten Ziele.
\item
  \textbf{Fazit} Zusammenfassung und Bewertung der Arbeit sowie
  Vorstellen von Erweiterungsmöglichkeiten.
\end{enumerate}
