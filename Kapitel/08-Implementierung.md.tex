\chapter{Implementierung}\label{implementierung}

\blindtext[1]

\section{Entwicklungsumgebung}\label{entwicklungsumgebung}

\begin{itemize}
\tightlist
\item
  Gitlab, Tickets, CI, etc.
\item
  Heroku
\end{itemize}

\section{Verwendete
Software-Komponenten}\label{verwendete-software-komponenten}

\begin{itemize}
\tightlist
\item
  Ionic/Quasar: Macht Sinn, wenn \emph{eine} App im Browser und mobil
  laufen muss. Hier gibt es zwei alleinstehende Apps, die völlig
  verschiedene Dinge tun
\end{itemize}

\section{Ausgewählte
Implementierungsdetails}\label{ausgewuxe4hlte-implementierungsdetails}

\subsection{Cool \#1}\label{cool-1}

\subsection{Cool \#2}\label{cool-2}

\subsection{Integration von xnes/snes9x.js und
vue.js}\label{integration-von-xnessnes9x.js-und-vue.js}

Drei notwendige Schritte:

\begin{enumerate}
\def\labelenumi{\arabic{enumi}.}
\tightlist
\item
  Deaktivieren von eslint für Kompilierungsergebnis snes9x.js,
\item
  Webpack-Konfiguration:
  \texttt{node:\ \{fs:\ \textquotesingle{}empty\textquotesingle{}\}},
\item
  Anfügen von \texttt{export\ \{Module\ as\ default\}} ans Ende von
  snes9x.js, damit die Datei korrekt von Webpack erkannt wird.
\end{enumerate}

\section{Vollständigkeit}\label{vollstuxe4ndigkeit}
