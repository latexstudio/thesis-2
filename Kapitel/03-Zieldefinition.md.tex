\chapter{Spezifikation}\label{spezifikation}

Dieser Abschnitt definiert die zu erfüllenden System-Anforderungen,
deren Erreichen im späteren Verlauf dieser Arbeit während der Evaluation
überprüft werden. Zunächst wird definiert, über welche Grundfunktionen
das System verfügen soll. Weiter gelistet werden die Leistungs- und
Qualitätsanforderungen sowie die Randbedingungen, innerhalb derer die
definierten Anforderungen gelten. Abgeschlossen wird das Kapitel durch
eine Auflistung der Kriterien, die explizit \emph{nicht} Gegenstand
dieser Arbeit sind.

\newpage

\todo{Kurze Einordnung}\todo{Völlig neu? Alt?}\todo{Einfache Lösung? Total schwer?}\todo{Forschung? Anwendung?}

Die \gls{snes}-Konsole unterstützt grundsätzlich nur \emph{lokalen
Multiplayer} mit bis zu zwei --- beziehungsweise 4, mit speziellem
Adapter --- Spielern. Die Eingabe der Befehle erfolgt über mitgelieferte
Eingabegeräte \gls{controller}, die jeweils über ein Kabel mit der
Konsole verbunden sind. Über einen Multiplayer-Modus zum Spielen über
das Internet, wie er von heutigen Konsolen unterstützt wird, verfügt die
Konsole nicht.

Ursprünglich für diesen Anwendungszweck nicht vorgesehen, wird im Zuge
der Arbeit ein System entwickelt, das das Spielen von \gls{snes}-Spielen
über das Internet ermöglicht. Die einzelnen Spieler sind dabei nicht
über eine Konsole, sondern über einen Webbrowser miteinander verbunden
(siehe Abbildung \ref{fig:spec_overview_1}). Die Verbindung wird dabei
über eine Web-Plattform, nachfolgend auch Web-Applikation (kurz
Web-App), hergestellt und verwaltet.

\myfig{Diagramme/Zieldefinition_0_Controller}
   {width=0.6\textwidth}
   {Bla foo bar.}
   {Home town flower}
   {fig:spec_overview_1}

Die große Herausforderung dieser Arbeit besteht darin, das \gls{snes} um
einen neuen Multiplayer-Modus zu erweitern, der das gemeinsame Spielen
übers Netzwerk überhaupt erst ermöglicht. Bereits bestehende
Emulations-Programme lösen das Problem in Ansätzen, in der Praxis ist
deren Verwendung aber unnötig komplex und benötigt neben dem
eigentlichen Spiel weitere Software und eine spezielle
Netzwerk-Konfiguration (siehe Abschnitt \todo{}).

Um die Grundfunktionen des zu erstellenden Systems zu nutzen, wird außer
einem aktuellen Desktop-Browser keine weitere Software benötigt. Das
Spiel wird direkt im Browser ausgeführt, wobei die Steuer-Befehle über
die jeweils angeschlossene Computer-Tastatur entgegen genommen und
verarbeitet werden.

Alternativ haben Benutzer die Möglichkeit zur Installation einer nativen
Mobile-App für iPhones und Android-Geräte, die neben der eigentlichen
Web-App entwickelt wird. Nach dem Koppeln der beiden Applikationen
erfolgt die Eingabe-Verarbeitung nicht mehr über die Computer-Tastatur,
sondern über den in der App angezeigten virtuellen \gls{controller}, der
einem originalen SNES-Controller nachempfunden ist. Das Prinzip ist in
Abbildung \ref{fig:spec_overview_2} dargestellt.

\myfig{Diagramme/Zieldefinition_1_Controller}
   {width=0.6\textwidth}
   {Bei Verwendung der Controller-App erfolgt die Eingabe der Steuerbefehle nicht mehr über den Web-Browser, sondern über das mit dem Web-Browser gekoppelte Smartphone.}
   {Controller-App übernimmt die Eingabe der Steuerbefehle}
   {fig:spec_overview_2}

Neben ihrer Funktion als ``besserer Controller'' schafft die
Controller-App außerdem die Möglichkeit, dem laufenden Spiel weitere
(lokale) Spieler hinzuzufügen. Das bedeutet, das pro Browser mehrere
Spieler vorhanden sein können, die sich einen Bildschirm teilen. Das
Hinzufügen der lokalen Spieler funktioniert funktioniert dabei so wie
die Kopplung des Smartphones von Spieler 1 und muss vom Spielleiter
initiiert werden (siehe Abbildung \ref{fig:spec_overview_3}. Durch diese
Flexibilität sind verschiedene Spiel-Konfigurationen möglich.

\myfig{Diagramme/Zieldefinition_2_Controller}
   {width=0.6\textwidth}
   {This flower was photographed at my home town in 2010.}
   {Home town Test}
   {fig:spec_overview_3}

\section{Funktionale Anforderungen}\label{funktionale-anforderungen}

Dieser Abschnitt beschreibt die Spezifikation des Soll-Zustands des zu
erstellenden Systems und seiner Komponenten. Der Soll-Zustand wird
während der Evaluation überprüft, in dem ein Vergleich zwischen Soll-
und Ist-Zustand vorgenommen wird.

Das zu entwickelnde System besteht aus den Komponenten \textbf{Web-App}
und \textbf{Controller-App}, deren Anforderungen nachfolgend aufgelistet
sind.

\subsection{Web-Applikation}\label{web-applikation}

Die Web-Applikation besteht aus vier verschiedenen Ansichten, die
untereinander verknüpft und zwischen denen der Benutzer während der
Nutzung navigiert:

\begin{itemize}
\tightlist
\item
  \textbf{Startseite},
\item
  \textbf{Authentifikation},
\item
  \textbf{Lobby},
\item
  und \textbf{Match}.
\end{itemize}

Abbildung \ref{fig:spec_views_and_flow} gibt einen Überblick über die
verschiedenen Ansichten und Navigationspfade der Web-Applikation. Die in
den jeweiligen Ansichten enthaltenen Funktionen werden nachfolgend
erläutert und bilden zusammen die funktionalen Anforderungen, die an die
Web-Applikation gestellt werden.

\myfig{Diagramme/Flow}
    {width=0.8\textwidth}
    {Ansichten und Navigationspfade der Web-Applikation.}
    {Cool}
    {fig:spec_views_and_flow}

\subsubsection*{Ansicht 1: Startseite}

Die Startseite ist eine keine wirkliche Ansicht. Die Funktion der
Startseite ist lediglich die Weiterleitung des Benutzers auf eine
nachfolgende Ansicht. Ist der Benutzer dem System bekannt, erfolgt eine
Weiterleitung auf die Lobby-Ansicht. Ist der Benutzer unbekannt, ist
eine Authentifikation notwendig und der Benutzer wird auf zur
Authentifikation-Ansicht weitergeleitet.

\subsubsection*{Ansicht 2: Authentifikation}

Voraussetzung für die Nutzung der Web-Applikation ist die
Authentifikation des Benutzers. Dabei wird der Benutzer aufgefordert
einen eindeutigen Benutzernamen auszuwählen. Die Eingabe erfolgt über
ein herkömmliches Web-Formular, das die Benutzereingaben validiert und
sicherstellt, dass ein eindeutiger Name gewählt wird. Nach der
Verarbeitung des Benutzernamens wird dieser im Browser des Nutzers
gespeichert und die Web-Applikation wechselt zur Lobby-Ansicht.

\subsubsection*{Ansicht 3: Lobby}

Nach der Authentifikation des Benutzers und dessen Weiterleitung auf die
Lobby-Ansicht werden dem Benutzer folgende Funktionen zur Verfügung
gestellt:

\begin{description}
    \item [Auflisten der aktuellen Matches] Nach der Authentifikation wird der Benutzer auf die Lobby-Ansicht weitergeleitet. Die Lobby ist eine Liste aller Matches, die von anderen Benutzern erstellt worden sind und denen der Benutzer beitreten kann, um an Matches teilzunehmen. Aufgelistet werden nur jene Matches, deren maximale Spielerzahl noch nicht erreicht ist. Die Liste zeigt den aktuellen Stand der Matches in Echtzeit an und wird automatisch aktualisiert, ohne dass das Browser-Fenster neu geladen werden muss.
    \item [Erstellen (“Hosten”) eines neuen Matches] Der Benutzer muss die Möglichkeit haben, ein neues Match zu erstellen, um entweder allein oder gemeinsam mit anderen Personen zu spielen. Voraussetzung zum Erstellen eines neuen Matches ist die Bereitstellung des Spiels in Form einer \acrshort{rom}-Datei, die vom Benutzer im Verlauf der Spielerstellung über ein Web-Formular hochgeladen werden muss. Neben der \acrshort{rom}-Datei muss ein Titel für das Match vergeben und die maximal gewünschte Spielerzahl festgelegt werden. Unterstützt werden sollen pro Match ein bis maximal vier Spieler. Die ursprünglich für das Spiel vorgesehene maximale Spielerzahl des hochgeladenen Spiels wird nicht ermittelt. Es liegt im Ermessen des Benutzers eine für das jeweilige Spiel sinnvolle\footnote{Zum Beispiel zwei für “Street Fighter 2” und vier für “Super Bomberman”.} maximale Spielerzahl festzulegen. Solange das Match existiert und der Benutzer das Spiel durch Schließen des Browser-Fensters nicht verlässt, hat er die Rolle des Spielleiters. Als solches stehen ihm für das erstellte Match mehr Funktionen zur Verfügung als den Benutzern, die dem Spiel als Mitspieler beitreten. Die Beschreibung der Funktionen befindet sich in Abschnitt \nameref{sec:view-match}.
    \item [Match beitreten] Der Benutzer soll einem bestehenden Match beitreten können, um daran teilzunehmen. Vor dem Beitreten werden keine Informationen abgefragt, das jeweilige Match muss nur ausgewählt werden. Grundsätzlich können alle Benutzer jedem gelisteten Match beitreten. Liegt die vom Spielleiter konfigurierte maximale Spielerzahl über der vom Spiel unterstützten, betritt der Benutzer das Match im Zuschauer-Modus und kann nicht ins Spielgeschehen eingreifen. Vor dem Betreten wird dem Benutzer nicht angezeigt, in welchem Modus (als Mitspieler oder als Zuschauer) das Match betreten wird.
\end{description}
\subsubsection*{Ansicht 3: Match}\label{sec:view-match}

Die Match-Ansicht setzt sich aus mehreren Bereichen zusammen, über die
unterschiedliche Funktionen zugänglich gemacht werden.

Die \textbf{Leinwand} ist der prominente Bereich der Ansicht, in dem das
laufende Spiel dargestelllt wird.

\begin{description}
    \item [Spiel anhalten, pauseren und neustarten] Der Spielleiter kann das Spiel über Schaltflächen starten, pausieren und neustarten.
    \item [Spielen im Vollbildmodus] Jeder Mitspieler hat hier die Option, die Leinwand über eine Schaltfläche in den Vollbildmodus umzuschalten. Der Vollbildmodus kann über die ESC-Taste verlassen werden.
\end{description}

Die \textbf{Spielerliste} listet alle im Match befindlichen Benutzer
auf. Die Liste wird stets aktualisiert und zeigt jedem Benutzer in
Echtzeit die aktuellen Mitspieler an.

\begin{description}
    \item [Entfernen eines Spielers aus einem Match] Der Spielleiter kann beliebige Spieler über eine Schaltfläche aus dem Match entfernen (“kicken”). Nach dem Entfernen wird der Benutzer auf die Lobby-Ansicht weitergeleitet. Eine Sperrzeit ist nicht vorgesehen. Der Spieler kann dem Match theoretisch erneut beitreten.
\end{description}

Den letzten Bereich der Match-Ansicht bildet der \textbf{Chat},
bestehend aus dem Chat-Verlauf und einem Formular zum Senden einer neuen
Chat-Nachricht. Der Chat-Verlauf wird nicht persistiert: Jedem Benutzer
werden nur diejenigen Nachrichten angezeigt, die nach dem Zeitpunkt
seines Beitritts zum Match verfasst worden sind.

\begin{description}
    \item [Match-eigener Chat] Über den Chat-Bereich hat jeder Benutzer die Möglichkeit, Nachrichten mit den Mitspielern aus dem Match auszutauschen. Chat-Nachrichten werden über ein Formularfeld eingegeben und über eine Schaltfläche abgesendet. Der Chat-Verlauf aktualisiert sich in Echtzeit. Alle im Match befindlichen Benutzer (inkl. Zuschauer) können am Chat teilnehmen und bekommen alle Nachrichten in Echtzeit angezeigt.
\end{description}

Wenn ein Benutzer ein Match durch Schließen des Browserfensters oder
durch die Navigation auf eine andere Seite verlässt, wird die
Spielerliste bei allen Mitspielern live aktualisiert. Verlässt der
Spielleiter das Match, wird das Match beendet und alle Mitspieler werden
auf die Lobby-Ansicht weitergeleitet.

\newpage

\todo{Gehört hier nicht hin}{Voraussetzung für ein netzwerkbasiertes Multiplayer-Spiel ist eine aktive Verbindung zwischen allen Mitspielern eines Spiels.
Wie diese Verbindung zustande kommt, ist von System zu System unterschiedlich. Bei den meisten Emulatoren ist es notwendig, einen Spieler als Spielleiter zu bestimmen. Das Emulator-Programm dieses Spielers wird dabei im Host-Modus gestartet und bietet einen Netzwerkdienst an, der über das Internet erreichbar sein muss und deren Adresse den Mitspielern mitgeteilt werden muss. Dieses Verfahren hat mehrere Nachteile, die bereits in Kapitel \todo{} erläutert wurden. Eine bessere Lösung, die sich mittlerweile auch durchgesetzt hat, ist das Konzept einer Spiel-Lobby, in dem alle laufenden Spiele aufgelistet sind. Die Lobbys verfügen meist über Such- und Filterfunktionen zum Finden des gesuchten Spiels. Die Lobby ist eine Lösung für das *Matchmaking* und erleichtert das, in dem es die Komplexität des Vorgangs, die im erstgenannten Ansatz vorhanden ist (IP-Adressen, Netzwerk-Konfiguration, Firewalls, etc.), vor den Spielern verbirgt.}

\subsection{Controller-Applikation}\label{controller-applikation}

\todo{}

\subsection{Weitere Anforderungen}\label{weitere-anforderungen}

\todo{Unterstützte Spiele}

\section{Nichtfunktionale
Anforderungen}\label{nichtfunktionale-anforderungen}

\todo{}

\subsection{Leistung- \&
Qualitätsanforderungen}\label{leistung--qualituxe4tsanforderungen}

\todo[inline]{Frames und Latenz} \todo[inline]{im LAN, nicht Internet}

\subsection{Anforderungen an die
Umgebung}\label{anforderungen-an-die-umgebung}

Alle definierten Anforderungen gelten nur im Kontext der hier
beschriebenen Umgebung --- die Lauffähigkeit des entwickelten Systems
ist ebenfalls nur dann gegeben.

\subsubsection{Nutzungsbedingungen}\label{nutzungsbedingungen}

Grundsätzlich unterstützt das System alle modernen Desktop-Browser, die
folgende Standards vollständig und fehlerfrei unterstützen:

\begin{enumerate}
\def\labelenumi{\arabic{enumi}.}
\tightlist
\item
  JavaScript
\item
  \gls{canvas}-Element
\item
  WebGL
\item
  WebRTC (inkl. RTCPeerConnection, MediaCapture)
\item
  WebAssembly
\item
  WebAudio
\item
  WebSockets
\end{enumerate}

\todo{Glossar-Links}

Installierte
Browser-Erweiterungen\footnote{Zum Beispiel \glqq{}uBlock Origin Extra\grqq{}, das diverse WebRTC-Funktionen deaktiviert.},
die die o.g. Technologien blockieren, deaktvieren oder die
Funktionsweise einschränken, müssen deaktiviert sein. Theoretisch
erfüllt und deswegen angestrebt ist eine Unterstützung der
Desktop-Varianten der Web-Browser Google Chrome und Mozilla Firefox
\todo{warum erfüllen die beiden das? Kompatibilitätsmatrizen, iswebrtcreadyyet}
--- unter den genannten Voraussetzungen. Die Browser verfügen meist über
unterschiedliche Implementationen der genannten Standards, wobei es
große Unterschiede bzgl. Funktionsumfang und Korrektheit geben kann.
Eine qualitativ gleichwertige Kompatiblität für mehrere Browser zu
schaffen ist kein triviales Problem, zu dessen Lösung innerhalb dieser
Arbeit keine Zeit zur Verfügung steht. Im Zuge der Umsetzung wird ein
Prototyp implementiert, der es möglich macht, sich einen ersten Eindruck
zur Kompatibilität zu verschaffen. Das System wird anschließend auf
diejenigen Web-Browser hin optimiert, die bei einem manuellen Vergleich
die höchste Kompatibilität haben und die Software standardkonform
ausführen.

Mobile Web-Browser (wie z. B. \glqq{}Mobile Safari\grqq{}) werden
grundsätzlich nicht unterstützt, da die entsprechenden Standards noch
nicht implementiert und integriert worden sind\todo{Datum?}.

\subsubsection{Betriebsbedingungen}\label{betriebsbedingungen}

Das entwickelte System ist eine komplette Web-Anwendung (\glqq{}Full
Stack\grqq{}) und benötigt neben verschiedenen Software-Bibliotheken
keine weiteren Module, die außerhalb des Kontextes dieser Arbeit
entstanden
sind\todo{Meine eigenen, evtl. vorher entwickelten. Third-Party natürlich.}.
Das Aufbauen auf bestehenden Teillösungen wird angestrebt und die
Integration derer dokumentiert. Voraussetzung für den Betrieb ist ein
Linux-Betriebssystem mit Unterstützung für \gls{docker}. Alle benötigten
Software-Pakete werden automatisch innerhalb der Docker-Umgebung
installiert und konfiguriert.

Das System ist nicht auf maximale Skalierbarkeit hin optimiert, weil
dies außerhalb des Kontextes dieser Arbeit liegt. Anfragen kleinerer
Benutzergruppen von 10-20
\todo{Nach Evaluation mit tatsächlicher Zahl ersetzen}gleichzeitigen
Nutzern, die gegebenenfalls während einer Testphase verarbeitet werden
müssen, sollen sich nicht negativ auf die beschriebene Qualität
auswirken.

\subsection{Randbedingungen \&
Abgrenzungskriterien}\label{randbedingungen-abgrenzungskriterien}

\begin{itemize}
\tightlist
\item
  Keine Benutzerkonten.
\item
  Keine Lag-Kompensation.
\item
  Eine Optimierung der \gls{accessibility} der Web-Applikation ist aus
  Zeitgründen nicht möglich.
\end{itemize}
