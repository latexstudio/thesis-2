\chapter{Spezifikation}\label{spezifikation}

Dieses Kapitel beinhaltet die Problembeschreibung und die Motivation,
die Anstoß zur Bearbeitung dieses Themas war. Der Hauptteil besteht aus
einer Zieldefinition der zu erstellenden Software. Dies beinhaltet auch
Aspekte, die ausdrücklich nicht Teil dieser Arbeit sind.

Dieser Abschnitt definiert die zu erfüllenden System-Anforderungen, die
im späteren Verlauf dieser Arbeit während der Evaluation überprüft
werden. Zunächst wird definiert, welche Grundfunktionen das System haben
soll. Weiter gelistet sind die Leistungs- und Qualitätsanforderungen
sowie die Randbedingungen, innerhalb derer die definierten Anforderungen
gelten. Abgeschlossen wird das Kapitel durch eine Auflistung von
Kriterien, die nicht Gegenstand dieser Arbeit sind.

\newpage

Das Ziel dieser Masterarbeit ist die Entwicklung einer Online-Plattform,
die es den Nutzern einfach macht, gemeinsam über das Netzwerk
miteinander SNES-Spiele zu spielen. Dafür existieren
Emulations-Programme in Hülle und Fülle. Warum muss man das neu machen?

\myfig{Abbildungen/Diagramme/Zieldefinition_0_Controller}
   {width=0.9\textwidth}
   {This flower was photographed at my home town in 2010.}
   {Home town flower}
   {fig:flower}

\myfig{Abbildungen/Diagramme/Zieldefinition_1_Controller}
   {width=0.9\textwidth}
   {This flower was photographed at my home town in 2010.}
   {Home town flower}
   {fig:flower}

\myfig{Abbildungen/Diagramme/Zieldefinition_2_Controller}
   {width=0.9\textwidth}
   {This flower was photographed at my home town in 2010.}
   {Home town Test}
   {fig:flower}

\begin{itemize}
\tightlist
\item
  Völlig neu? Alt?
\item
  Einfache Lösung? Total schwer?
\item
  Forschung? Anwendung?
\end{itemize}

\section{Funktionsumfang}\label{funktionsumfang}

\subsection{Muss-Kriterien}\label{muss-kriterien}

\subsection{Kann-Kriterien}\label{kann-kriterien}

\subsection{Abgrenzungskriterien}\label{abgrenzungskriterien}

\section{Bedienung}\label{bedienung}

\missingfigure{Flow-Diagramm (Sketch)}

\section{Anforderungen an die
Umgebung}\label{anforderungen-an-die-umgebung}

\section{Leistung- \&
Qualitätsanforderungen}\label{leistung--qualituxe4tsanforderungen}

\section{Betriebs- \&
Nutzungsbedingungen}\label{betriebs--nutzungsbedingungen}

\section{Randbedingungen}\label{randbedingungen}
