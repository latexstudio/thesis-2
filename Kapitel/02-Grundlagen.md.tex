\chapter{Grundlagen}\label{grundlagen}

Inhalt dieses Kapitels ist eine Einführung in die technischen Grundlagen
dieser Arbeit. Es werden die wichtigsten Aspekte und Technologien
erläutert und für das Lösungsumfeld essentielle Begriffe erklärt. Die
Erläuterungen schaffen ein Grundverständnis für die in den nachfolgenden
Kapiteln behandelten Aspekte und sind für eine Beurteilung der
entwickelten Lösung notwendig. Schwerpunkte sind für die Zielsetzung
potentiell geeignete Web-Technologien, Emulatoren \& ROMs und
Grundlagenthemen zur Kommunikation in Computer-Netzwerken.

\newpage

\section{Emulation von
Konsolenspielen}\label{emulation-von-konsolenspielen}

Wie eingangs erwähnt, sind zum Spielen von \gls{snes}-Spielen am
Computer zwei Dinge notwendig: Ein Emulationsprogramm --- auch
\gls{emulator} genannt --- und das jeweilige Konsolenspiel, das
ausgeführt werden soll. Das Spiel liegt dabei in Form einer Datei vor.
Sie beinhaltet die exakt gleichen Daten, die sich für gewöhnlich auf den
Modulen befinden, die zum Spielen in die Konsolen gesteckt werden
müssen. Zum Kopieren der Daten existieren spezielle Geräte, mit denen
eine solche Datei erzeugt werden kann. Dabei wird ein genaues Abbild des
Moduls erzeugt, inklusive der Sektoren und der jeweiligen
Dateisystem-Struktur. Die Dateien werden oft als ROMs bezeichnet, da die
Daten des Spiels innerhalb des Moduls in einem ROM-Chip gespeichert
sind. Um ein Spiel auszuführen zu können, muss die Spiel-Datei in den
Emulator geladen werden.

Der Emulator ist ein Software-Programm, das es ermöglicht Konsolentitel
auf einem Computer auszuführen. Dies funktioniert, in dem die Hardware
der jeweiligen Konsole nachgeahmt wird. Die verschiedenen Komponenten
der Konsole (CPU, Soundchip, etc.) sind dabei so in Software
implementiert dass sie sich möglichst so verhalten, wie die
nachzuahmende Hardware. Erst eine möglichst exakte Emulation der
ursprünglichen Laufzeitumgebung ermöglicht die Ausführung von
unveränderten Spielen in Form von ROM-Dateien. Die Spiele, bzw. das
Abbild der Software, ist eine exakte Kopie des Originals.

\begin{itemize}
\tightlist
\item
  Roms
\item
  Emulatoren
\end{itemize}

\section{Der Multiplayer-Modus}\label{der-multiplayer-modus}

\todo[inline]{Im Netzwerk! Im Gegensatz zum lokalen Multiplayer.}
\todo[inline]{Welche Eigenschaften/Merkmale hat der lokale Multiplayer, die auch im Netzwerk-Multiplayer gelten sollen}

Das erste entwickete Computerspiel war ein Multiplayer-Spiel und
benötigte damit mindestens zwei Spieler, die gemeinsam an einem Gerät
spielen. Die Verarbeitung der Benutzereingaben erfolgt zentral an dem
einzigen Gerät, so wie auch die Darstellung der grafischen Oberfläche.

\section{Grundbegriffe der
Netzwerk-Kommunikation}\label{grundbegriffe-der-netzwerk-kommunikation}

\todo[inline]{Evtl. ganz an den Anfang schieben, damit man im Multiplayer-Abschnitt darauf verweisen kann.}

\subsection{Latenz}\label{latenz}

\begin{itemize}
\tightlist
\item
  Minimalste Latenz essentiell beim Spielen
\item
  Alles über \unit[N]{ms} ist nicht gut genug. Um angepeilten
  Maximalwert zu erreichen, System entsprechend konstruieren und
  Technologien wählen, die eine hinreichende Lösung erst möglich machen.
\item
  Wo von hängt Latenz ab? Was kann man da überhaupt machen?
\item
  RTT, TCP vs.~UDP?
\end{itemize}

\subsection{RTT}\label{rtt}

\subsection{Realtime}\label{realtime}

\section{Das Realtime-Web}\label{das-realtime-web}

Vergleich zur alten Herangehensweise. Frontend viel Logik, Backends
eigentlich nur noch API.

\subsection{Websockets}\label{websockets}

\subsection{HTTP/2}\label{http2}

\subsection{SSE}\label{sse}

\subsection{WebRTC}\label{webrtc}

\todo[inline]{Hier nur beschreiben. Hier kann ich letztlich auch nur raten,
dass die gewählten Technologien sinnvoll sein werden. Ob die getroffene Auswahl gut war (wurden Ziele erreicht?), zeichnet sich ggf. schon während der Implementierung ab. 
Ergebnisse und Erkenntnisse, die die Bildung eines ersten Urteils ermöglichen, werden erst in der Evaluationsphase geschaffen.}

\todo[inline]{NICHT: Vergleichen/Abwägen, Entscheiden, Begründen}
